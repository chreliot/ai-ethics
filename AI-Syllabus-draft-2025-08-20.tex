\documentclass[11pt]{article}
\usepackage{url}                                                                                                \usepackage[left=.8in,right=.8in,top=.8in,bottom=.8in]{geometry}
\usepackage{amssymb}
\usepackage{enumitem}
\usepackage{fancyhdr} \pagestyle{fancy}
\setlength\headheight{14pt}
\fancyhead{} \fancyhead[LE,RO]{PHI 162 syllabus, Fall \the\year{}}
\usepackage{sectsty} % \allsectionsfont{\sffamily}
\usepackage{tikz} \usetikzlibrary{calendar}

\usepackage{multicol}

\usepackage{fontspec} \defaultfontfeatures{Ligatures={Common}, Mapping=tex-text, Numbers=Lining}
\usepackage{xunicode}% provides unicode character macros

\setromanfont[Scale=1.0]{Adobe Caslon Pro}
\setsansfont[Scale=1.0]{Helvetica}
\setmonofont[Scale=1.0]{Courier}

% \setdefaultleftmargin{1em}{}{}{}{}{}

\usepackage[
pdftitle={AI Ethics draft syllabus},
pdfauthor={Christopher H. Eliot},
pdfkeywords={artificial intelligence, ethics, Hofstra, course, syllabus},
pdflang={English}
]{hyperref}


\begin{document}
\thispagestyle{plain}
        \begin{center}
                \textsf{{\Large Philosophy of Biology} \\ PHI 162 (Fall \the\year)}
        \end{center}
\bigskip

\noindent \begin{tabular}{ll}
	Instructor: & \href{https://www.hofstra.edu/faculty/fac_profiles.cfm?id=415}{Dr.~Christopher Eliot, Associate Professor} of \href{http://www.hofstra.edu/academics/colleges/hclas/phi/}{Philosophy} \\
Office:  & 104F \href{https://map.concept3d.com/?id=19#!m/24116}{Heger Hall} (516-463-4516)\\
Office hours: & Mon 3–4 (104F Heger and Zoom);  Tue 10–11 (Zoom); and by appointment\\
Email:  & \href{mailto:Christopher.H.Eliot@hofstra.edu}{\url{Christopher.H.Eliot@hofstra.edu}} \\ 
Texts:  & available on the course's Canvas page or in class\\
Course time/place:  & Mon+Wed 11:20–12:45 in \href{http://www.hofstra.edu/About/InfoCenter/info_intmap.html}{Roosevelt 0206} \\
% & Section 2 (91922): Mon, Fri 11:15–12:40 in \href{http://www.hofstra.edu/About/InfoCenter/info_intmap.html}{Davison 17} \\
\end{tabular}

% \subsection*{Course website:}

% The course website is \href{https://chreliot.github.io/phi162/}{https://chreliot.github.io/phi162/}

\subsection*{Important dates:}

\begin{tabular}{l l}

	First writing assignment due:	 & Oct 2 \\
	First test/midterm exam: & Oct 16 \\
	% Optional draft deadline for second essay: & April 10 \\
	Second writing assignment due:	 &  Nov 13 \\
	% Third short essay due:	 & May 8 \\
	Second test/final exam:	 & Dec 18, 8–10 am in our classroom\\
\end{tabular}

% \input{SundayCalendar} % Change the dates below to reflect the schedule above.
% 
% \begin{center}
% \begin{tikzpicture}[every calendar/.style={
%         month label above centered,
%         month text={{\textbf{\%mt}}},
%         if={(Sunday) [blue!70]},
% 	    if={(Saturday) [blue!70]},
%         week list sunday,
%     }]
%     \matrix[column sep=1em, row sep=1em] {
%         \calendar(feb)[dates=2022-02-01 to 2022-02-last]; &
%         \calendar(mar)[dates=2022-03-01 to 2022-03-last]; &
%         \calendar(apr)[dates=2022-04-01 to 2022-04-last]; &
%         \calendar(may)[dates=2022-05-01 to 2022-05-last]; \\
%     };
% 	\draw[thick](mar-2022-03-02) circle (7pt);
% 	\draw[thick](mar-2022-03-16) circle (7pt);
% 	\draw[thick](apr-2022-04-27) circle (7pt);
% 	\draw[thick](may-2022-05-18) circle (7pt);
% \end{tikzpicture}
% \end{center}


\subsection*{What we're doing here:}

Philosophy of Science is a core area of contemporary philosophy. It arises from three kinds of  questions that philosophers have worked on since long before science existed: First, How do we know what we know? This is called “Epistemology.” Second, What is the nature of reality, and what exists? This is called “Metaphysics” or “Ontology.” And third, How does one reason well or poorly? This is “Logic.” With the growth of science in recent centuries, philosophers have paid close attention to what science has to say about the answers to these questions and how it does so (or could do so), including how scientists do or should reason.

Philosophy of Biology asks these questions about Biology, the science of life and living things. Biology has implications for a whole range of discussions in philosophy, for our understanding of nature, of knowledge, and of ourselves as biological beings. We can also use philosophical strategies to analyze it as a science, including its justification, scope, and limits.

We will start out with a brief overview of pre-Darwinian frameworks for Biology and then at Darwin's explanatory framework for biology. Then we will consider major debates about evolutionary biology and its implications. After this initial unit, we will immerse in a series of current discussions in philosophy of biology, selected partly in relation to student votes.


\subsection*{This course's main goals are:}

\begin{enumerate}[noitemsep]

	\item  to familiarize you with how philosophical questions arise about biology, what some specific ones are, and what philosophical ideas have been offered in response to them;

	\item to enhance your ability to formulate questions and analyze arguments about scientific ideas, in a general way and with respect to specific (putative instances of it);

	\item to enhance your ability to express and defend ideas through arguments, in speech and writing;

    \item to enhance your ability to recognize philosophical problems, and to familiarize you with philosophical strategies for approaching them;

	\item to enhance the sophistication of your understanding of biology and philosophy.

\end{enumerate}

\subsection*{Course requirements:}
% \emph{By remaining in this course, you acknowledge that that you
% agree to and understand these requirements.}

\begin{enumerate}

    \item You will need to \textbf{access the \href{https://chreliot.github.io/phi162/}{course website}} regularly to check in on assignments and other regularly-updated course information.
	\item You will need to \textbf{access the assigned texts} online, figure out your strategy for reading them, and read them.
	% \item You need to \textbf{bring the texts} to class.
	\item You are expected to \textbf{attend} the class sessions. Your on-time attendance rate will influence your participation grade. % I will be taking attendance at each session. % In lecture, please say hello or wave at me \emph{before} the lecture begins, on Tuesdays if you're in section H3 and Thursdays if you're in section H2. I will normally be up front. Arriving late counts as half an absence. Absences beyond 2 will reduce your grade by .15/4. 
	Since you will be responsible on tests for knowing what happened in class even if you did not attend, it may be helpful to figure out someone in the class to ask. I can name what topics we discussed in class, but I can't repeat the lecture or discussion or send them to you by email. I don't normally have notes that would make sense to you. Others should.
	\item You are expected to \textbf{read}, and \textbf{be prepared to discuss}, the assigned texts. You will receive a participation grade which figures in both your contributions to class and the apparent preparedness reflected in them. Participation is not mere attendance, and involves volunteering contributions. I may even call on you, and expect you to have something relevant to say. Though presentations will help explain the readings, you cannot expect to understand our  discussion without reading yourself. Much of the point of our meeting in person is that you bring you be able to engage \emph{your independent scholarship} with that of other scholars—your peers and me. Obviously, that requires having both read and digested the text. Do it.
	\item \textbf{Electronic devices} may be used during class to engage with assigned readings and to make notes on the class, and only for those purposes. I may ask that they be put away for parts of the course (except for students who have accommodations for them).
\item You will need to attend and take a \textbf{first/midterm test} and a \textbf{second test/final exam} at the end of the term. The final will be comprehensive, but will emphasize the second half of the course. For both tests I will circulate a list of questions in advance, from which the actual test will be drawn. There will therefore be no questions on the tests which you have not seen in advance.  The second test will cover the entire course but will strongly emphasize the second half. No bathroom breaks will be permitted. Make-up tests will not be permitted except with a note from a Hofstra official or doctor. Make your travel plans accordingly.
\item There will be two out-of-class \textbf{writing assignments}. The topics will be handed out about two weeks before the due dates, which are listed on the course schedule. They will follow guidelines in the handout “Writing a Philosophy Paper,” which will be posted on the course's Blackboard site. Further, detailed information about essay-grading criteria is also contained therein.
	\item Finally, I always hope this goes without saying and am sometimes disheartened to find that it doesn't: I ask you to \textbf{be respectful} of the business of the class during class sessions. Let's have fun, but also respect the challenge others face in maintaining attention. 
	% In this course it is acceptable drink or even eat quietly during class. But please do not distract yourself and others by watching videos, loudly chomping Doritos, etc. 
	Especially as this will be a small discussion course, I expect engagement from you. That requires not sleeping and not engaging in distracting side-conversations, but also being sensitive to the need to make room for as many students' contributions as possible.
\end{enumerate}

\subsection*{Evaluation:}

I would prefer to teach entirely without grades, but their existence has various kinds of value for you. So we have them. Here is where they come from:

To the degree it is practically feasible, I evaluate your work anonymously, to eliminate unconscious biases and approach objectivity. (Obviously, under certain circumstances, it is not practically feasible.) Your grades will be calculated according to the University's standards, relative to course expectations, and relative to other members of the class. This does \textit{not} mean your grade will be ``curved'' to a mean score, but it may be adjusted upwards depending on overall class performance. That is, you will earn at least the score you deserve according to University standards, but also one related to how other students performed. Writing assignments and exams will be graded using a 4-point scale. ``Incomplete'' status will also not be given automatically, nor in the absence of a compelling, written request.

Your grade will be calculated as follows:\\

\begin{tabular}{ll}
	Participation & 20\%  \\
	First Essay   & 18\%  \\
	Second Essay  & 22\%  \\
	First Test    & 18\%  \\
    Second Test   & 22\%  \\
\end{tabular}
\medskip

 \noindent Grades for writing assignments, quizzes, and tests will be converted to Hofstra's 4-point scale before calculation:

 \medskip
 \begin{tabular}{l}
 4 = \textbf{A} represents exceptional work.\\
 3 = \textbf{B} represents superior work.\\
 2 = \textbf{C} represents satisfactory work.\\
 1 = \textbf{D} represents below-satisfactory work.\\
 \end{tabular}


% \noindent Your participation grade is significant component of your final grade, and it reflects the degree to which you offer \emph{both} attention and  prepared contributions to our discussions. You will receive a participation grade for roughly each third of the course. Each one will count for 6\% of your grade. I do this to let you know in advance how I think you're doing at participation, and to give you the opportunity to improve, if necessary.
% You receive a seminar grade every day of class: 0, 1, or 2. It has two components. The first is minimal and could not be easier: you receive 1 point if you attend class and have arrived by the time it begins, and a 0 if not. Then, the second component reflects your preparation, i.e.~that you have read the assignment for that day \emph{and} demonstrate that preparation by offering \emph{informed} contributions to the discussion. You receive 1 point for doing so, and 0 if not. Arriving on time, having prepared the assignment, and offering contributions nets you 2 points. At the end of the term, your participation grade will be converted to a 4-point scale simply by doubling your average.


\subsection*{Instructor's own academic honesty policy:}

Artificial intelligence tools may not be used for submitted work without explicit permission.

Representing someone else's work as your own, or any other form of scholastic dishonesty (as defined by the University), will automatically earn you an F for the course, beyond the required report. So this is the key point you should internalize now: If you ever find yourself in circumstances where it seems like a good idea to be dishonest, please come talk to me about what we can do about the circumstances instead. You will find that while after I've detected scholastic dishonesty, the outcome is severe and automatic, \emph{beforehand} I try to be as helpful as possible.

\subsection*{Syllabus adjustments:} 

Unexpected events can lead to changes in the schedule and syllabi. However, it's important to me that you not feel that the rules have changed on you mid-stream, so, if any changes are necessary, I will make them as fair as possible.

\subsection*{Contacting me:} 

The best way to contact me is by email; I check it regularly. If you have a quick question, often we can take care of it by email. But if you need more help, don't hesitate to visit my office hours, listed above, or ask to meet. Speak to me after class or email me to see if we can set up a mutually convenient time. Phone messages are not usually a reliable to reach me.

\subsection*{Honors option requirements:} 

\begin{enumerate}
		\item Read five additional articles or chapters over the course of the term, each providing commentary on the issues introduced by the standard readings and class discussion. These are be drawn from primarily from writing by contemporary philosophers of biology. Likely topics and readings are below.
\item Meet with me individually five times over the course of the term, once each for each additional reading, for discussion of that article or chapter. Students are expected to arrive having read the article.
\item For each supplementary reading assignment, write a very short summary of the main argument of the article and raise at least one substantive question, and possibly also clarifying questions. These response papers will be at least one page, and will need to be completed before the meetings.
\item Answer supplementary questions on the course's final exam, based on the supplementary material.
\end{enumerate}

\subsection*{General university policies:}

Information about Academic Dishonesty; Disability Accommodations; Resources for Students who are Pregnant; Temporary Adjustments/Academic Leave of Absence; Deadlines and Grading Policies; Discrimination, Harassment, Sexual Misconduct; Absences for Religious Observance and specific policies relating to COVID-19 guidelines including mask wearing, campus closures/snow days, class attendance, and class seating is available on the Provost's webpage at the link below.

 \href{https://www.hofstra.edu/about/administration/provost/provost-information-for-students.html}{https://www.hofstra.edu/about/administration/provost/provost-information-for-students.html}


\subsection*{Final thought:}

Inevitably, grades are a function of performance, not of effort in itself. I can't reasonably assess effort. What's challenging varies. You will need to figure out what \emph{you} need to do to perform well. I will try to help you with what's hard for you, if I know you need help. In the end, you are responsible for your education, however, and if you are confused, you should ask a question, or I may assume you understand. Unless you discuss them with me in person or by email, I will also likely not be aware of any dissatisfactions you have with any aspect of the course. I hope you will not be dissatisfied. I think this material is fun and useful, and I believe an important part of my job is trying to show you why it is.	




\end{document}
