\documentclass[letter,11pt]{article}

\usepackage[left=.8in,right=.8in,top=.8in,bottom=.8in]{geometry}               
\usepackage{amssymb}
\usepackage[flushleft]{paralist}
\usepackage{fancyhdr} \pagestyle{fancy} 
\setlength\headheight{14pt} 
\fancyhead{} \fancyhead[LE,RO]{PHI 000 syllabus, Fall 0000{}}
\usepackage{sectsty} % \allsectionsfont{\sffamily}
\usepackage{tikz} \usetikzlibrary{calendar}
\usepackage{url}                                                                
\usepackage{multicol}


\usepackage{fontspec} \defaultfontfeatures{Ligatures={Common}, Mapping=tex-text, Numbers=Lining}
% \usepackage{xunicode}% provides unicode character macros 
\setromanfont[Scale=1.0]{Adobe Caslon Pro}
\setsansfont[Scale=1.0]{Helvetica}
\setmonofont[Scale=1.0]{Courier}

\setdefaultleftmargin{1em}{}{}{}{}{}

\usepackage[
pdftitle={PHI 014 Syllabus},
pdfauthor={Christopher H. Eliot},
pdfkeywords={Philosophy, Ethics, Hofstra, course, class, syllabus},
% pdflang={English}
]{hyperref}


\begin{document}
\thispagestyle{plain}
        \begin{center}
            \textbf{\Large Artificial Intelligence in Society: \\ New Challenges, Ethical Futures} 
        \end{center}
\bigskip

\noindent \begin{tabular}{ll}
Professor: &  \\
Office:  &  \\
Office hours: &  \\
Email:  &   \\
% (\href{mailto:Christopher.H.Eliot@hofstra.edu}{Christopher.H.Eliot@hofstra.edu} also works)\\
Texts:  & readings will linked or distributed through a Canvas reading list\\
% & Blackburn, \href{http://www.worldcat.org/title/being-good-an-introduction-to-ethics/oclc/464980786}{\emph{Being Good: A Short Introduction to Ethics}, 2nd ed.} \\

Course time \& place:  &  \\
% & temporary Sec.~7: Tue \& Thu 1:00–2:25 (Zoom) \\
% & Sec.~4: Mon \& Wed 12:50–2:15 in \href{http://www.hofstra.edu/About/InfoCenter/info_intmap.html}{Heger 101}
% Course site: & \href{https://chreliot.github.io/phi14/}{https://chreliot.github.io/phi14/} 

\end{tabular}

\subsection*{What we're doing here:}

Emerging artificial intelligence technologies create new challenges for traditional ideas about human ethics. “Ethics,” here, means the study of what to do—including what is good, bad, right, and wrong—and why we should do one thing rather than another. By exploring how artificial intelligence creates specific challenges you will learn how to think about possible solutions. We will do that by bringing traditional ways of thinking about ideas like right action, goodness, personal character, rights and responsibilities, and fairness and justice to confront a world being changed by artificial intelligence. We will observe how these ideas interact with artificial intelligence in specific cases and practice reasoning logically about what to do with AI, using those concepts. We will explore where our society might need new ethical ideas, and we will think about how to build a promising future society we will want to live in, that includes AI.

Philosophical Ethics, which offers an intellectual backbone for this course, is a core area of investigation in English-speaking (“anglophone”) Philosophy. One way of understanding what “Philosophy” is is analysis of questions that cannot be entirely answered empirically (which means “through observation”). Because answers to philosophical questions cannot be proved by observation alone, philosophers use and analyze \emph{arguments} to gain understanding. In this course we will be examining important and influential arguments about artificial intelligence and also developing some arguments of our own. We will try to understand a variety of possible positions we could come to adopt for ourselves, and some of their strengths and weaknesses.

% This course has four main goals. They are:
% 
% \medskip
% \begin{compactenum}
	% \item to familiarize you with some major historical and contemporary positions in normative and metaethics;
	% \item to enhance your ability to formulate questions and analyze arguments, especially about values;
	% \item to enhance your ability to express and defend ideas through arguments, in speech and writing;
	% \item to encourage the development and sophistication of your understanding of ethics and morality. 
% \end{compactenum}


% \subsection*{Dates you are responsible for:}
% 
% \smallskip
% \begin{compactitem}[]
% \item Complete the first writing assignment: multiple dates, see site
% \item Submit the second writing by upload to Blackboard: Apr 21
% \item Take the first (midterm) test: Mar 24
% \item Take the second (final) test: May 19 10:30–12:30
% \end{compactitem}
% 
% \medskip
% \input{SundayCalendar}
% \begin{center}
% \begin{tikzpicture}[every calendar/.style={
%         month label above centered,
%         month text={{\textbf{\%mt}}},
%         if={(Sunday) [blue!70]},
% 	    if={(Saturday) [blue!70]},
%         week list sunday,
%     }]
%     % \matrix[column sep=1em, row sep=1em] {
%     %     \calendar(sep)[dates=2017-09-01 to 2017-09-last]; &
%     %     \calendar(oct)[dates=2017-10-01 to 2017-10-last]; &
%     %     \calendar(nov)[dates=2017-11-01 to 2017-11-last]; &
%     %     \calendar(dec)[dates=2017-12-01 to 2017-12-last]; \\
%     \matrix[column sep=1em, row sep=1em] {
%         \calendar(feb)[dates=2021-02-01 to 2021-02-last]; &
%         \calendar(mar)[dates=2021-03-01 to 2021-03-last]; &
%         \calendar(apr)[dates=2021-04-01 to 2021-04-last]; &
%         \calendar(may)[dates=2021-05-01 to 2021-05-last]; \\
%     };
% 	\draw[thick](mar-2021-03-24) circle (7pt);
% 	\draw[thick](apr-2021-04-21) circle (7pt);
% 	\draw[thick](may-2021-05-19) circle (7pt);
% 	% \draw[thick](dec-2017-12-15) circle (7pt);
% 	% \draw[thick](dec-2017-12-20) circle (7pt);
% \end{tikzpicture}
% \end{center}

\subsection*{Important dates:}

\begin{tabular}{l l}
	First quiz: & Sep 29 \\
	First test: & Oct 15 \\
	Second quiz: & Nov 12 \\
	% Optional draft deadline for second essay: & April 10 \\
	Second  test: & Dec 15 1:30–3:30 \\
	% Third short essay due:	 & May 8 \\
	% Third test:	 & May 17 \\
\end{tabular}

% \smallskip
% \noindent \emph{See the course site for other important dates and current information about the above.}

\subsection*{Course requirements:}
% \emph{By remaining in this course, you acknowledge that that you agree to and understand these requirements.}

\begin{enumerate}
	\item You will need to \textbf{check Canvas} for reading assignments, writing assignments, and other essential information. It serves as an extension of this syllabus and our dynamic schedule. Probably bookmark it. 
	\item You will need to \textbf{access the texts} which will be linked from our uploaded to Canvas and figure out your strategy for reading them: print, tablet, screen?
	\item You need to \textbf{bring the texts} (or in some way have them readily available to you) for class.
	\item You will be expected to \textbf{read} and \textbf{be prepared to discuss} the assigned texts. You will receive a participation grade which figures in both your contributions to class and the apparent preparedness reflected in them. Participation is not mere attendance; it involves volunteering contributions. I may even call on you and expect you to have something relevant to say. Though presentations will help explain the readings, you cannot expect to understand our  discussion without reading yourself. Much of the point of our meeting synchronously is that you bring you be able to engage \emph{your independent scholarship} with that of other scholars—your student peers and me. Obviously, that requires having both read and digested the text. Do it.
	\item You will need to \textbf{participate} in our class discussion by asking and responding to questions. See the Participation Rubric on Canvas to understand how participation will be assessed.	
	\item You will submit six \textbf{short writing assignments} through Gradescope. Instructions for accessing Gradescope will be posted to Canvas.
	\item There will be two \textbf{quizzes}, handwritten in class. You will need to bring a writing implement. Each of the quizzes will involve writing similar to what you will have done in your out of class assignments.
	\item You will need to attend and take a \textbf{first/midterm test} and a \textbf{second test/final exam} at the end of the term. The second test will be comprehensive, but will emphasize the second half of the course. More information about the test content will be posted to Canvas. For each test, I will circulate a list of questions in advance, from which the actual test will be drawn. There will therefore be no questions on the tests which you have not seen in advance.  The second test will cover the entire course but will strongly emphasize the second half. No bathroom breaks will be permitted during tests. Make-up tests will not be permitted except when you are formally excused. Make travel plans accordingly.
	\item You are expected to \textbf{attend} the class sessions. I will record attendance. % In lecture, please say hello or wave at me \emph{before} the lecture begins, on Tuesdays if you're in section H3 and Thursdays if you're in section H2. I will normally be up front. 
% Arriving late counts as half an absence. Absences beyond 2 will reduce your grade by .2/4 
More than 2 sessions of missed attendance will be factored into your participation grade. You \emph{do not} need to contact me about being absent! However, if you have doctors' notes or excuses from Hofstra officials, keep them on file until the end of the semester, in case you need to show why you had more than 2 absences. Also, since you will be responsible on tests for knowing what happened in class even if you did not attend, it may be helpful to figure out someone in the class you can ask. I can tell you what topic we discussed in class, but I can't repeat the lecture or discussion or email notes. I don't have notes that would make sense to you. Others might.
	% \item
	% Laptop and tablet use is allowed in C\&E \emph{lectures} at the discretion of the lecturer.
\item \textbf{Electronic devices} will be permitted during some portions of class sessions, during presentations. During other portions of class sessions, you will be asked to put away all devices. 

% \item For each class, you'll write a very brief \textbf{reaction paper}---one per week. It should be one to two pages long. One full page will be completely adequate, and you shouldn't feel any pressure to write more than that. It should reflect on at least two of the writings, demonstrating your thoughtful engagement with them, and should raise at least one substantive question. It may also raise clarifying questions. You should print bring it with you to class to turn-in \emph{or} submit it to Blackboard via upload, under the Assignments button.
% \item There will be two out-of-class \textbf{writing assignments}. Full instructions will be posted on the course site.
\item Finally, I always hope this goes without saying, but: I ask you to \textbf{respect} the business of the class and of other students' efforts to focus. Respect doesn't always mean solemn silence. See the participation rubric on the Canvas site for more details.
\end{enumerate}

\subsection*{Evaluation:}

I would prefer to teach entirely without grades, but their existence has various kinds of value for you. So we have them. Here is where yours come from:

To the degree it is practically feasible, I evaluate your work anonymously, to eliminate unconscious biases and approach objectivity. (Obviously, under certain circumstances, it is not practically feasible.) Your grades will be calculated according to the University's standards, relative to course expectations, and relative to other members of the class. This does \textit{not} mean your grade will be ``curved'' to a mean score, but it may be adjusted upwards depending on overall class performance. That is, you will earn at least the score you deserve according to University standards, but also one related to how other students performed. Writing assignments and exams will be graded using a 4-point scale. ``Incomplete'' status will also not be given automatically, nor in the absence of a compelling, written request.

\noindent Your grade will be calculated as follows:
\begin{tabular}{ll}
Participation & 24\%  \\
	% Reaction papers & 25\% \\
Out-of-class assignments & 24\%  \\
% Second writing & 20\%  \\	
First quiz & 4\% \\
Second quiz & 4\% \\
First test  & 20\%  \\
Second test  & 24\%  \\
\end{tabular}

\medskip
\noindent I will circulate partial/preview grades for participation at the one third and two thirds marks, so you have a clear sense of how I think you're doing. The participation rubric is posted on the Canvas. Grades for writing assignments, quizzes, and tests will be converted to Hofstra's 4-point scale before calculation.

\medskip
\begin{tabular}{l}
4 = \textbf{A} represents exceptional work.\\
3 = \textbf{B} represents superior work.\\
2 = \textbf{C} represents satisfactory work.\\
1 = \textbf{D} represents below-satisfactory work.\\
\end{tabular}
\medskip
% \noindent Your participation grade is significant component of your final grade, and it reflects the degree to which you offer \emph{both} attention and  prepared contributions to our discussions. You will receive a participation grade for roughly each third of the course. Each one will count for 6\% of your grade. I do this to let you know in advance how I think you're doing at participation, and to give you the opportunity to improve, if necessary.
% You receive a seminar grade every day of class: 0, 1, or 2. It has two components. The first is minimal and could not be easier: you receive 1 point if you attend class and have arrived by the time it begins, and a 0 if not. Then, the second component reflects your preparation, i.e.~that you have read the assignment for that day \emph{and} demonstrate that preparation by offering \emph{informed} contributions to the discussion. You receive 1 point for doing so, and 0 if not. Arriving on time, having prepared the assignment, and offering contributions nets you 2 points. At the end of the term, your participation grade will be converted to a 4-point scale simply by doubling your average.

\subsection*{Academic integrity policy:}

Familiarize yourself with the university's Academic Integrity policy. Representing someone else's work as your own, or any other form of academic dishonesty (as defined by the University), will automatically earn you an F for the course, beyond the required report to the University. So this is the key point you should internalize now: If you ever find yourself in circumstances where it seems like a good idea to be dishonest, please come talk to me about what we can do about the circumstances instead. You will find that while \emph{after} I've detected academic dishonesty, the outcome is severe and automatic, \emph{beforehand} I try to be as helpful as possible.

\subsection*{Generative artificial intelligence policy:}

This course does not have a simple, across-the-board rule against the use of generative artificial intelligence (including ChatGPT and other large language models, among other software) in your coursework. The course presumes that, like spelling-correction software and search engines, generative AI is a widely used tool, and that as with all tools, students benefit from understanding their strengths and limitations.

Students should consider that here, just like in the broader world, using generative AI but not understanding its limitations can produce serious, negative consequences. Specifically, students using generative AI must be mindful of the following:
\begin{enumerate}
\item You are paying here, in part, for help with developing understanding and skills. If you use generative AI in a way that undermines your learning, you are not getting what you paid for and you are wasting your time.
\item When using assistive technologies, you remain responsible for everything you turn in. By turning something in, you assert authorship of it, no matter what tools you have used to write it.
\item This course's academic integrity policy does not stop applying just because you've used generative AI. Submitting false citations, for example, is an academic integrity violation. As of 2025, ChatGPT and other LLMs generate false citations \emph{frequently}. 
\item For the purposes of this course, including text generated by generative AI in work for which you assert authorship is not on its own an academic integrity violation. However, if the text you submit happens to use a source without appropriate attribution or required quotation marks, that \emph{is} a clear academic integrity violation.
\item A significant portion of the graded work for this course will not be submitted electronically, but rather handwritten in class, and you will not be able to use generative AI to produce it. If, when you are taking those assessments you haven't developed skills and understanding because you relied on generative AI earlier, these asssessments will probably not go well for you. And our goal—mine and presumably yours—is for it to go well!
\end{enumerate}	

\subsection*{Syllabus adjustments:} 

The schedule may dynamically update on Canvas. Beyond that, events can lead to changes in the schedule and syllabi. However, it's important to me that you not feel that the rules have changed on you mid-stream, so, if any changes are necessary, I will make them as fair as possible.

\subsection*{Contacting me:} 

The best way to contact me is by email; I check it regularly. If you have a quick question, often we can take care of it by email. But if you need more help, don't hesitate to visit my office hours, listed above, or ask (in person or by email) to set up another mutually-convenient time, in person or online. Phoning is not a reliable way to reach me.

\subsection*{General university policies applying to this course:}

Information about Class Attendance; Recording in Classes; Inclement Weather; Academic Integrity; Student Health and Wellness, Disability-Related Accommodations; Resources for Students who are Pregnant; Temporary Accommodations; Academic Leave of Absence; Absences for Religious Observance; University Deadlines; Grading Policy; and Sex-Discrimination, Sexual Assault, Dating and Domestic Violence, and Stalking is available on the Provost’s webpage at the following URL:\\ \url{https://www.hofstra.edu/provost/policies-wording-syllabi.html}.

% \subsection*{About the readings:}
%
% \begin{compactitem}[\(\triangleright\)]
% 	\item Read the reading assignments before the date of their Discussion section. It is generally a good idea to read them before their associated lectures.
% 	\item The readings will be in the book you purchased or posted on the Course Documents section of the course's Blackboard site. To access it, visit \href{http://my.hofstra.edu/cp/home/loginf}{\url{my.hofstra.edu}}, login, and click on the Blackboard tab, and then on the link for this course. You can go to a computer lab at the University and print them out there if you don't have a printer. Or, just on the days with electronic readings, you may bring a laptop with the readings on it.
% 	\item Things on this schedule will happen on the day indicated, or later. All the listed readings will be used in the course in that order, so you can read ahead without risk.
% 	% \item You are not always required to read the whole of the works listed, but which sections you should read will be determined by the professor lecturing, and not all of them are specified on the schedule yet.
% 	\item In some cases, I will announce in class (and/or on Blackboard) that you should, for our purposes, focus your attention on particular sections of the reading.
% 	\item Remember that these readings aren't magazine articles; you're going to have to be a little patient and focus your attention.
%     \item I strongly recommend writing while you read---if not in your text because you want to preserve its resale value, then on a separate page. It helps focus your attention, and leaves you with starting points for contributing in class.
% 	\item Work on figuring out as you read what are the parts to pay closest attention to and what to pay less-close attention to. That academic skill will take you far.
% 	\item I would never discourage you from looking up information and academic work that \emph{complements} your reading. That's what scholars do. But I feel strongly about the value to be had from direct, uncontaminated, experiential first-engagement with the texts, and adamantly discourage you from shortcutting that, for your sake. Note also the related demands of academic honesty.
%     % \item Guide your reading by thinking through, or ideally writing down, answers to the potential quiz-questions listed above.
% 	\item Discussing the reading with someone else even before you come to class can make it easier to understand. We never learn anything so well as when we try to explain it out loud to someone else.
%  	% \item The readings marked with an asterisk (*) are more difficult---allocate extra time.
% 	\item Changes are possible because unexpected things happen; any changes will be announced in class, on Blackboard's announcements, and on your Hofstra email account. I hope there won't be any.
% 	% \item Errors are possible on the schedule; please let me know if you find any.
% \end{compactitem}

\subsection*{Department's learning goals and objectives for this course:}

% The following, as adopted by the Philosophy Department, apply to this course:

	% \smallskip
	% \footnotesize
	\noindent \textbf{Goals for Philosophy courses:}

	\begin{compactitem}[\(\triangleright\)]
	   \item    Students understand major philosophical ideas accurately
	   \item    Students apply their understanding of ideas in novel contexts
	   \item    Students write effectively
	   \item    Students speak effectively
	   \item    Students argue with precision, balance, and insight
	   \item    Students read analytically, critically, and empathetically
	   \item    Students critically assess their own commitments and ideas
	\end{compactitem}   
\medskip

	\noindent  \textbf{Specific learning objectives for Philosophy courses:}

	\begin{compactitem}[\(\triangleright\)]
	   \item    Objective 1a: Students give accurate and relevant answers, complete with supporting details, to specific questions about philosophical ideas relevant to the course.
	   \item    Objective 1b: Students give accurate accounts of philosophical ideas relevant to the course in the context of criticizing or assessing those ideas.
	   \item    Objective 2a: Students speculate, in well-informed, well-supported, and plausible fashion, about what a given philosopher would say about a novel issue or problem.
	   \item    Objective 2b: Students extrapolate creatively and plausibly from their knowledge of philosophers or philosophical positions in developing their own related ideas.
	   \item    Objective 3a: Students write paragraphs that exhibit clarity, focus, a good command of the subject matter, and an orderly development of ideas.
	   \item    Objective 3b: Students write multi-paragraph pieces that exhibit clarity, focus, a good command of the subject matter and an ability to work with that subject matter creatively, and an orderly development of ideas both within and across paragraphs.
	   \item    Objective 4a: Students speak in clear, focused, well-informed, and orderly fashion.
	   \item    Objective 5a: Students state arguments accurately and clearly, and identify strengths and weaknesses of different arguments.
	   \item    Objective 5b: Students develop and defend their own arguments, taking into account a variety of philosophical positions but adding original insights or emphases.
	   \item    Objective 7a: Students explain difficult passages clearly, accurately, and thoroughly.
	   \item    Objective 7b: Students use apt quotations and creative, critical, plausible readings of texts in their writing.
	   \item    Objective 8a: Students are able to explain the weaknesses of their own present positions, and the strengths of competing positions.
	   \item    Objective 8b: Students are able to explain why their pre-theoretical commitments have or have not changed as a result of what they have learned in the course, and if they have changed how they have done so. 
	\end{compactitem}
\normalsize

% \subsection*{Course content outline:}
% 
% \begin{multicols}{2}
% 
% \begin{compactenum}
% \item Arguments and reasoning
% \item Plato and the problem of justice
% \item Seven challenges to ethics
% \begin{compactitem}[\(\triangleright\)]
% \item Redundancy on religion
% \item Relativism
% \item Psychological Egoism
% \item Eliminative Naturalism
% \item Determinism
% \item False Consciousness
% \item Empiricism
% \end{compactitem}
% \item Kantianism
% \begin{compactitem}[\(\triangleright\)]
% \item Kant
% \item Criticisms of Kantianism
% \end{compactitem}
% \item Utilitarianism
% \begin{compactitem}[\(\triangleright\)]
% \item Mill
% \item Criticisms of Utilitarianism
% \item Applied Utilitarianism
% \end{compactitem}
% \item Virtue theory
% \begin{compactitem}[\(\triangleright\)]
% \item Aristotle
% \item Criticisms of Virtue Ethics
% \end{compactitem}
% \item Conventionalism
% \end{compactenum}

% \end{multicols}

\subsection*{Final thought:}

Inevitably, grades are a function of performance, not of effort in itself. I can't reasonably assess effort. What's challenging varies from student to student. You will need to figure out what \emph{you} need to do to perform well. I will try to help you with what's hard for you, if I know you need help. In the end, you are responsible for your education, however, and if you are confused, you should ask a question, or I will assume you understand. Unless you discuss them with me in person or by email, I will also likely not be aware of any dissatisfactions you have with any aspect of the course. I hope you will not be dissatisfied. I think this material is fun and useful, and believe an important part of my job is trying to show you why it is.	


\end{document}