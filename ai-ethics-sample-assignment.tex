\documentclass[12pt]{article}

\title{Example writing assignment}
\author{Christopher Eliot}
\date{2025-11-04}

\usepackage[
pdftitle={AI Ethics Example Writing Assignment},
pdfauthor={Christopher H. Eliot},
pdfkeywords={writing,essay,assignment,philosophy,AI,Hofstra},
pdflang={English}
]{hyperref}

\usepackage{enumitem} % units,nicefrac}
\usepackage[left=1in,right=1in,top=1in,bottom=1in,driver=luatex]{geometry}

\usepackage{fontspec}
% \defaultfontfeatures{Mapping=tex-text}
\setromanfont[Mapping=tex-text]{Adobe Caslon Pro}
\setsansfont[Scale=MatchLowercase,Mapping=tex-text]{Helvetica}
\setmonofont[Scale=MatchLowercase]{Inconsolata}

\pagestyle{empty}

\begin{document}

\begin{center}
        \textbf{Writing Assignment} \\ PHI 40, Spring 2027 (Eliot)
\end{center}

\bigskip

\noindent


\noindent In response to the following, write two paragraphs, with each paragraph containing at least 200 words.

\subsection*{Topic}

\textbf{First paragraph:} Emrys Westacott suggests that machines can be fairer than people. But Kevin Macnish argues that AI-automated CCTV surveillance that involves humans in specific ways improves fairness, in comparison with both human and fully-automated-AI surveillance. How, according to Macnish, can partially AI-automated surveillance, involving a human observer, be fairer than fully-automated surveillance? What does he claim in support of that? In explaining that support, quote and cite one sentence or clause from Macnish that supports that he indeed claims what you say he claims about fairness.

\medskip
\noindent \textbf{Second paragraph:} Pointing to a specific claim that Macnish makes in support of his conclusion, offer a reason or reason to doubt that claim, such that partly-AI-automated surveillance would not be the fairest option. Use that doubt to make a brief case that either (a) fully-automated or (b) fully-human CCTV surveillance (choose one of those) would actually be fairer than partly-automated surveillance. In explaining your case, quote and cite one sentence or clause from Macnish that supports that he indeed claims what you say he claims.

\subsection*{Style}

Write as though you are addressing someone who is not in the class and does not have a background in philosophy. Do your best to explain the ideas concisely and precisely. Use a brief, illustrative example if that might help explain an idea. 

Make sure to present arguments clearly, showing (1) what is serving as a premise or supporting reason, (2) what is the conclusion, and (3) how reasoning connects them. Also remember that a criticism of an argument cannot simply describe a different perspective or offer support for a different conclusion; instead, it must argue that there is a problem with the target argument. The two main kinds of problems with arguments we have discussed in class are problems with logic or problems with the truth of premises.

A few additional remarks about style are in the “Writing Philosophy” handout on Canvas.

\subsection*{Submission}

Follow the link to this assignment, \#0, under “Assignments” on the course Canvas site. Upload your writing assignment to Gradescope by pasting your paragraphs into the text box on the linked page. While it is possible to type your answers directly into that text box, that is very risky, because a browser crash or myriad other mishaps could delete your work. I recommend writing in a separate text editor or word processor. 

 % This is not a research paper, and researching will not contribute significantly to your grade.
% 
% Submit your assignment by uploading it through the First Assignment upload link under the \texttt{Assignments} button on the course's Blackboard site. Submit your assignment online by May 17. If you want to have your assignment grade posted by the day before your final exam, the (optional) deadline for uploading your assignment is May 10. You do not need to submit a copy on paper. On the assignment itself, type your name only \emph{at the end} of the document rather than at the beginning, to facilitate anonymous reading and grading. Titling your assignment is optional. You might just put the course title, the date, and something like "Second Assignment" at the top of the document.
% 
% \begin{enumerate}
% \item Lay out clearly how different basic assumptions lead Root and Spencer to different conclusions about whether races exist. That is, what differences in their assumptions lead to differences in their conclusions? Then, in at least the final \nicefrac{1}{3}, imagine how each might be critical of the other's assumptions or reasoning.
% \item Recall Descartes's evil demon argument. In the first \nicefrac{1}{3} of your essay, develop an updated  example, perhaps a science-fiction scenario, which similarly casts doubt on whether we actually know what we think we know. Use that example to make an updated skeptical argument, paralleling Descartes's, for the conclusion that we can't have knowledge. Be clear what the premises and conclusion of that argument are. Then, in the remaining  \nicefrac{2}{3} of your essay, criticize that argument by adopting the perspective of either Plato or Hume.
% \item Imagine that Descartes and Hume are in class together and their professor says “You can't know anything.” Write a dialogue (in script format) between Descartes and Hume illustrating how each of them would respond, both to the professor and to each other's answers. Hint: they will each be mostly — but not entirely — critical of the professor and of each other. Focus on bringing out their reasoning about what basic claims support other claims.
% \item Imagine that any two of the following three are in class together (pick two): a hedonist and Epictetus and Susan Wolf. Their professor says, after class, “So how are you going to decide what to do with your life?” Consider how each would answer that question, and bring out that answer by using questions from the other. Focus on bringing out their reasoning about what basic claims support other claims. Then, bring out the differences in their positions by having them each ask the other challenging questions and raise objections about the other's views.
% \end{enumerate}

% The formats accepted are rich text (\texttt{.rtf}), plain text (\texttt{.txt}), Word (\texttt{.doc} or \texttt{.docx}), \texttt{.html}, and \texttt{.pdf}.
% Please name the file you submit ``1'' and then your last name. For example, if it were my paper, I would name the file \texttt{1eliot.doc} or \texttt{1eliot.rtf} (that is, with whatever the appropriate extension is); this saves me a hassle---thank you.

% Write as if you're addressing someone who is not in the class and does not have a background in philosophy. Remember to support your attribution of ideas to authors with brief, focused quotations and page references.

% \bigskip
%
% \noindent Think about how, based on ideas we've discussed in the course, the world could be a little different than how we normally think it is. In other words, think of an assumption that many of us hold that was challenged by a reading or discussion in the course --- whether or not you think that challenge was successful in the end. Write an essay beginning with “So, you think the way things are is that $X$?” where $X$ is a belief you think a lot of people probably hold. Here's an example: “So, you think the way things are is that science can't tell us what's right and wrong?” Then, in the rest of the first paragraph, describe some reasoning that your interlocutor might believe supports the claim in question. Then, in the remainder of the essay, develop and then evaluate an argument against this claim. In doing so, draw substantially on one text we looked at in the course after Darwin. For this question, write in the second person (using “you”). That is, ignore the rule against that from the “Writing” handout. Think about the project as if you're writing a newspaper editorial, for instance.
%
%
%


% \textbf{It's 1850. \emph{Origin of Species} hasn't yet been published. Only the other pre-Darwinian accounts of evolution (and anti-evolutionary views) are on the scientific landscape. Which of the following features do you want or not want in a scientific theory of evolution? Choose three to discuss. Argue for why a new theory of evolution should have or should not have each feature. Then, drawing on chapters 3, 4, and 14 of \emph{Origin}, argue why we should understand Darwin's new theory as having or not having the features you discuss. Here's the list to choose from:}
%
%
% \begin{enumerate}
% 	\item teleological
% 	\item theological
% 	\item tautological
% 	\item reflects contemporary culture
% 	\item experimental
% 	\item includes causal laws
% 	\item includes a mechanism
% 	\item accounts for origins of life
% 	\item includes progressive view of life
% 	\item anthropocentric
% \end{enumerate}

% And don't treat the questions as mere suggestions: if I ask you to spend \nicefrac{1}{3} of your essay on some issue, write at least 400 words on it; don't write a minimum-length paper and devote only \nicefrac{1}{5} of it to that issue.


% \begin{enumerate}
       % \item Darwin and Nietzsche each take a position on whether certain kinds of morality are natural or unnatural. Compare their arguments, bringing them into conversation with each other, and then (in at least the final \nicefrac{1}{4} or so) adjudicate how well each responds to the other's argument(s).

       % \item  Explain in your own words what reasons Plato's character Glaucon seems to offer in \emph{Republic} for the conclusion that we \emph{ought} to behave unjustly, at least under some particular circumstances. Employing brief, supporting quotations from the text, in the first \nicefrac{1}{2}  to \nicefrac{2}{3} lay out the argument in your own words. Then, in the final \nicefrac{1}{3} to \nicefrac{1}{2}, offer criticism of this argument---either of its premises or of its reasoning, or both.
%
% \item How does Aristotle argue for a teleological interpretation of organisms and/or their traits? Then, what is a way of understanding there being functions (as in “the function of the heart is to pump blood”) in post-Darwinian biology? Is the new understading of functions teleological; is it subject to the same criticisms Aristotle's theory was?
%
% \item Compare T.H. Morgan's genetic reasoning in part of either his 1910 or his 1911 paper or both. Then, Consider Goddard's genetic reasoning about there being a genetic basis for feeble-mindedness. How do they differ? Specifically, if we understand Morgan's reasoning as good reasoning, where do we find problems in Goddard's?



% \item For Darwin, is the state of nature for organisms competitive and individualistic or cooperative and mutualistic? Why or why not?
%
% \item What is adaptationism, and what is Lloyd's main argument against adaptationism (in the essay and video lecture)? Then, in at least the final \nicefrac{1}{3}, offer some potential critcicisms of her argument. Does she answer them, or could she?
%
%
%
%
% \item Contact me if there's something else directly connected with the course readings that you really want to write about.


		% \item In the first \nicefrac{1}{2} to \nicefrac{2}{3}, explain the difference(s) between relativism and antirealism (the view that there are no moral facts, of which emotivism is one form) as positions about morality. As you do so, discuss what reasons might motivate someone to believe in each of those positions, and observe how those reasons differ. Then, in the final \nicefrac{1}{3} to \nicefrac{1}{2}, explain whether (what is in your view) the strongest objection or objections to relativism also apply to subjectivism, and why they do or do not.

		% \item In the first \nicefrac{1}{2} to \nicefrac{2}{3}, explain the difference(s) between psychological egoism and the view that we ought to behave selfishly (let's call that ``moral egoism''). As you do so, discuss what arguments a person might offer that each of those positions is the \emph{right} one, and whether they are the same. Then, in the final \nicefrac{1}{3} to \nicefrac{1}{2}, explain how arguments against each of the two positions must differ from one another.

		% \item  In the first \nicefrac{1}{2} to \nicefrac{2}{3}, explain Blackburn's argument that evolutionary theory does not pose a threat to ethics. Then in the final \nicefrac{1}{3} to \nicefrac{1}{2},

      % \item Plato's character Glaucon, Joel Feinberg, Charles Darwin, and Friedrich Nietzsche each take positions on whether human beings are by nature selfish. Choosing any two of those, compare their arguments, bringing them into conversation with each other, and then (in at least the final \nicefrac{1}{4} or so) adjudicate how well each responds to the other's argument(s).
 	%
	%       \item In approximately the first \nicefrac{2}{3}, lay out and explain Ayer's argument for the conclusion at there is no fact of the matter about moral statements (like ``stealing is wrong'')---why, in other words, they aren't true. In the latter \nicefrac{1}{3}, offer some responses of your own to Ayer's argument, to its premises or its reasoning. Specifically, does he prove that there are no moral truths?
	%
	%    \item Lay out, and then in at least the final \nicefrac{1}{3} to \nicefrac{1}{2} argue against, Nietzsche's argument that behaving according to conventional (that is, ordinary) morality is part of a kind of `herd mentality' suitable only for weak and powerless people.
	%
	%
	% \item Examine where your values come from. Spend at most the first  \nicefrac{1}{3} of your essay doing that. Then, in the second \nicefrac{1}{3}, discuss what a moral relativist might say about whether your values are true. In the final \nicefrac{1}{3}, offer an argument against the relativist that your views are not merely relative. That is, explain why not.

% \end{enumerate}

% \subsection*{Group B:}
%
% \begin{enumerate}
%
%        \item Nietzsche's and Aristotle's arguments each rely on the idea of a great person. But they offer different ideas about what makes a person great. In the first \nicefrac{2}{3}, lay out their views of what makes a person great, offering some comparisons between the two. In the final \nicefrac{1}{3}, defend one or the other, or your own, original view.
%
%       % \item In approximately the first \nicefrac{2}{3}, lay out and explain Ayer's argument for the conclusion at there is no fact of the matter about moral statements (like ``stealing is wrong'')---why, in other words, they aren't true. In the latter \nicefrac{1}{3}, offer some responses of your own to Ayer's argument, to its premises or its reasoning. Specifically, does he prove that there are no moral truths?
%
%        \item In approximately the first \nicefrac{2}{3}, lay out and explain the problem Foot raises for Kantian ethics, and her proposed solution to that problem. In the final \nicefrac{1}{3}, offer some potential criticisms---perhaps, but not necessarily those Kant would offer---of her solution.
%
%        % \item Lay out, and then in at least the final \nicefrac{1}{3} to \nicefrac{1}{2} argue against, Nietzsche's argument that behaving according to conventional (that is, ordinary) morality is part of a kind of `herd mentality' suitable only for weak and powerless people.
%
%        % \item In approximately the first \nicefrac{2}{3}, lay out and explain Williams's main argument for the conclusion that there is something wrong with Utilitarianism. In the final \nicefrac{1}{3}, offer some responses of your own to Williams's argument. This might (though it need not necessarily) include a discussion of different versions of Utilitarianism, and how they might differently address Williams's problem.
%
%        \item Can it ever be immoral to harm your future self? As an example, consider recreationally using brain-damaging psychoactive drugs? Why or why not? Citing textual support, argue towards what you think (any two of) Kant's, Mill's or Aristotle's positions on this would be, and then in at least the final \nicefrac{1}{4} compare them with your own position.
%
%        \item Employing relevant positions of (any two of) Kant, Mill, and Aristotle, analyze the morality of the activity of prostitution. That is, for them and for you, is prostitution always wrong, or always acceptable, or something in between? Why? After analyzing what those philosophers would say, in at least the final \nicefrac{1}{3}, make an argument for one particular view.
%
%        % \item In approximately the first \nicefrac{2}{3}, lay out and explain the what argument Feinberg makes by using the example of Nowheresville. In the final \nicefrac{1}{3}, offer more than one potential criticism of it.
%
% 	\item Using (any two of) Kant, Mill, and Aristotle, consider whether there are reasons to think that it is always better to be morally perfect. Alternately, are there any arguments for the view that it might be better \emph{not} to be morally perfect? What would that mean, and how could it be better?

% \end{enumerate}

% \bigskip

% \noindent Submit your essay through Blackboard, as described above, on or before the next-to-last day of the semester, December 19. If you want to have your grade post by the night before the second test/exam, upload it by December 15. Beyond December 19, I won't be able to offer more time, since the semester is ending.


\end{document}
